\documentclass[sigconf]{acmart}
\usepackage{arydshln}

\usepackage{booktabs} % For formal tables


% Copyright
\setcopyright{none}
\settopmatter{printacmref=false}
%\setcopyright{acmcopyright}
%\setcopyright{acmlicensed}
%%\setcopyright{rightsretained}
%\setcopyright{usgov}
%\setcopyright{usgovmixed}
%\setcopyright{cagov}
%\setcopyright{cagovmixed}
\setlength{\belowcaptionskip}{-17pt}

%% DOI
%\acmDOI{10.475/123_4}
%
%% ISBN
%\acmISBN{123-4567-24-567/08/06}
%
%%:Conference
\acmConference[]{}{}{}
%\copyrightyear{2016}
%
%\acmArticle{4}
%\acmPrice{15.00}

% These commands are optional
%\acmBooktitle{Transactions of the ACM Woodstock conference}
%\editor{Jennifer B. Sartor}
%\editor{Theo D'Hondt}
%\editor{Wolfgang De Meuter}


\begin{document}
\title{ParSplice Keyspace Locality}
%\titlenote{Produces the permission block, and
%  copyright information}
%\subtitle{Extended Abstract}
%\subtitlenote{The full version of the author's guide is available as
%  \texttt{acmart.pdf} document}
%`
%`\author{Michael A. Sevilla, Carlos Maltzahn}
%`\affiliation{\institution{University of California, Santa Cruz}}
%`\email{{msevilla, carlosm}@soe.ucsc.edu}
%`
%`\makeatother
%`\author{Bradley W. Settlemyer, Danny Perez, David Rich, Galen M. Shipman (LANL)}
%`\email{{bws, danny_perez, dor, gshipman}@lanl.gov}
%`
%`\begin{abstract}
%`
%`Our analysis of the key-value activity generated by the ParSplice molecular
%`dynamics simulation demonstrates the need for a distributed, load balancing
%`key-value store.  We observe clear access regimes and hot spots that offer
%`significant opportunity for optimization. We leverage the Mantle load balancing
%`framework, which was originally designed for distributed file systems, to
%`dynamically switch policies and present a two policy scheme that achieves 96\%
%`efficiency while using only 7.6\% of the memory resources required by the base
%`case. Finally, we demonstrate how a machine learning clustering technique is an
%`effective method for detecting access patterns within the keyspace over time.  
%`
%`\end{abstract}

%
% The code below should be generated by the tool at
% http://dl.acm.org/ccs.cfm
% Please copy and paste the code instead of the example below. 
%
%\begin{CCSXML}
%<ccs2012>
% <concept>
%  <concept_id>10010520.10010553.10010562</concept_id>
%  <concept_desc>Computer systems organization~Embedded systems</concept_desc>
%  <concept_significance>500</concept_significance>
% </concept>
% <concept>
%  <concept_id>10010520.10010575.10010755</concept_id>
%  <concept_desc>Computer systems organization~Redundancy</concept_desc>
%  <concept_significance>300</concept_significance>
% </concept>
% <concept>
%  <concept_id>10010520.10010553.10010554</concept_id>
%  <concept_desc>Computer systems organization~Robotics</concept_desc>
%  <concept_significance>100</concept_significance>
% </concept>
% <concept>
%  <concept_id>10003033.10003083.10003095</concept_id>
%  <concept_desc>Networks~Network reliability</concept_desc>
%  <concept_significance>100</concept_significance>
% </concept>
%</ccs2012>  
%\end{CCSXML}

%\ccsdesc[500]{Computer systems organization~Embedded systems}
%\ccsdesc[300]{Computer systems organization~Redundancy}
%\ccsdesc{Computer systems organization~Robotics}
%\ccsdesc[100]{Networks~Network reliability}
%
%
%\keywords{ACM proceedings, \LaTeX, text tagging}


\maketitle

\section{Motivation}

The PDSW about ParSplice~\cite{perez:jctc20150parsplice} shows:
\begin{enumerate}
  \item keyspace is structured
  \item we don't need an unlimited cache
  \item dynamic policy works for a while
\end{enumerate}

The machine learning portion classifies read throughput into regimes. We could
like the machine learning to detect the superbasins.

\section{Keyspace Locality}


\section{Machine Learning Techniques}
\begin{itemize}
  \item K-means
  \item DBScane
  \item online learning
\end{itemize}

\section{Implementation}
\begin{itemize}
  \item HXHIM
\end{itemize}


\bibliographystyle{ACM-Reference-Format}
\bibliography{references} 

\end{document}
