\section{Introduction}

% What is the problem we are trying to solve
[DESCRIBE PROBLEM WE ARE TRYING TO SOLVE - CUT/PASTE FROM PDSW PAPER]

% What is Mantle
To solve this problem, we integrate Mantle into
ParSplice~\cite{perez:jctc20150parsplice}. Mantle is a dynamic policy engine
that decomposes complicated services into manageable pieces so developers
unfamiliar with a code-base can reason about its behavior. Administrators use
the framework to describe ``when" they want data moved, ``where" they want data
moved, and ``how much" of the data to move.  This abstraction helps developers
unfamiliar with the domain quickly reason about, develop, and deploy new
policies that control temporal and spatial locality. It has already proven to
be a critical control plane for improving file system metadata load
balancing~\cite{sevilla:sc15-mantle} and in this work we show its usefulness in
cache management.

% What are we going to show
We show how Mantle:
\begin{itemize}

  \item decomposes cache management into independent policies, making the
  problem more manageable and facilitating rapid development. We quickly implement
  the ``how much" policy and focus on the ``when" policy for the majority of this paper.

  \item has useful primitives that, while designed for file systems, are
  crucial for effective cache management. This finding shows how the Mantle
  engine generalizes to a different domain and code-base.

  \item can be used to quickly deploy a variety of cache management strategies,
  ranging from basic algorithms and heuristics to statistical models and machine
  learning. This mixing and matching may not outperform a hard-coded solution but
  can more adeptly change to different workloads and hardware. 

\end{itemize}

We demonstrate these contributions by changing the input to our molecular
dynamics simulation, which changes the keyspace access patterns. 

%First, we implement the ``how much" abstraction as an LRU and then we focus on
%the ``when" abstraction. 

%Manageable: abstracts away complexities of the system (pass around to others,
%use different strategies) 

