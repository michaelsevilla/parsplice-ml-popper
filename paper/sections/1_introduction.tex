\begin{abstract}

Our analysis of the key-value activity generated by the ParSplice molecular
dynamics simulation demonstrates the need for more complex cache management
strategies. Baseline measurements show clear keyspace access patterns and hot
spots that offer significant opportunity for optimization. We use the data
management and policy engine from the Mantle system to dynamically explore a
variety of techniques, ranging from basic algorithms and heuristics to
statistical models, calculus, and machine learning. While Mantle was originally
designed for distributed file systems, we show how the collection of
abstractions effectively decomposes the problem into manageable policies for a
different domain and service.  Our exploration of this space results in a
dynamically sized cache policy that, for our initial conditions, sacrifices
negligible performance while using only 28\% of the memory required by our
hand-tuned cache.

\end{abstract}

\section{Introduction}

The fine-grained data annotation capabilities provided by key-value storage is
a natural match for many types of scientific simulation. Simulations relying on
a mesh-based decomposition of a physical region may result in millions or
billions of mesh cells. Each cell contains materials, pressures, temperatures
and other characteristics that are required to accurately simulate phenomena of
interest. In our target application, the
ParSplice~\cite{perez:jctc20150parsplice} molecular dynamics simulation, a
hierarchy of caches and a single persistent key-value store are used to store
both observed minima across a molecule's equation of motion (EOM) and the
hundreds or thousands of partial trajectories calculated each second during a
parallel job. Unfortunately, if we scale the system the IO to the storage
hierarchy will quickly saturate both the storage and bandwidth capacity of a
single node, so we need more effective data management techniques, such as
cache management or load balancing across a cluster.

\begin{figure}[t]
\noindent\includegraphics[width=0.5\textwidth]{figures/cache-management.png}\\

\caption{Using our data management language and Mantle policy engine, we
designed a dynamically sized caching policy for the ParSplice application. The
solution uses knowledge about the application to detect key access patterns and
adjust the cache accordingly. 
\label{fig:cache-management}}
\end{figure}

In this paper, we design cache management policies for ParSplice, driven by a
detailed analysis of the key-value accesses over the course of a long running
simulation across a variety of initial conditions. The default ParSplice
implementation uses an unlimited sized cache, shown by the ``No Cache
Management" line Figure~\ref{fig:cache-management}. This solution is
unacceptable for HPC environments, where memory is precious and a common goal
is to keep memory for such data structures below 3\%\footnote{Empirically, we
find this threshold works well for most applications}.  While users can
configure ParSplice to resize the cache when it reaches a certain threshold,
this solution requires tuning and parameter sweeps; the dashed ``Cache (too
small)" curve in Figure~\ref{fig:cache-management} shows how a poorly
configured cache hurts performance.  Our final solution, shown by the
``Dynamically Sized Cache" line in Figure~\ref{fig:cache-management}, detects
keyspace access patterns and re-sizes the cache accordingly.  Without tuning or
parameter sweeps, our solution saves more memory than a hand-tuned cache
without sacrificing performance.  More importantly, it works for a variety
initial conditions without changing the policy itself.

% What is Mantle
To design more flexible cache management policies, like our ``Dynamically
Sized Cache" policy, we use the data management language and policy engine
from the Mantle paper~\cite{sevilla:sc15-mantle}. This framework allows us to
dynamically explore the effects of different software-defined cache management
strategies for the changing key-value workloads generated by ParSplice.  Mantle
was originally touted as a programmable file system metadata load balancer, but
we realize now that the collection of abstractions designed for file systems
was a control plane that improved metadata access. So in this paper we refer to
Mantle as a policy engine that injects policies written in our data management
language directly into a running service and show how this approach is useful
for reasoning about and designing different cache management strategies in
ParSplice.  By service, we mean a system that manages data and responds
requests, such as a file system or key-value store.  Developers write policies
for ``when" they want data moved and ``how much" of the data to move, then the
framework executes these policies whenever a decision needs to be made.  These
abstractions help developers unfamiliar with the domain quickly reason about,
develop, and deploy new policies that control temporal and spatial locality. We
show that Mantle:

\begin{itemize}

  \item decomposes cache management into independent policies that can be
  dynamically changed, making the problem more manageable and facilitating rapid
  development. Changing the policy in use is critical in applications such as
  ParSplice that have alternating stable and chaotic keyspace access patterns
  over the course of a long-running simulation.  

  \item can be used to quickly deploy a variety of cache management strategies,
  ranging from basic algorithms and heuristics to statistical models and machine
  learning.

  \item has useful primitives that, while designed for file system metadata
  load balancing, turn out to also be effective for cache management. This
  finding shows how the policy engine generalizes to different domains and
  enables control of policies by machines instead of administrators.

\end{itemize}

% this gives us many policies that are effective across disciplines
% - reuse: eases burden of writing policies
% - autonomic: lays groundwork for an adaptable policy that mixes/matches policies
% FUTURE WORK

This last contribution is explored in Sections~\S\ref{sec:arch-specific}
and~\S\ref{sec:dom-specific}, where we try a range of policies from different
disciplines; but more importantly, in Section~\S\ref{sec:scope}, we conclude
that the collection of policies we designed for ParSplice's cache management
are very similar to the policies used to load balance metadata in the Ceph file
system (CephFS) suggesting that there is potential for automatically adapting
and generating policies dynamically. 

%Manageable: abstracts away complexities of the system (pass around to others,
%use different strategies) 

