\section{Related Work}

Key-value storage organizations for scientific applications is a field gaining
rapid interest. In particular, the analysis of the ParSplice keyspace and the
development of an appropriate scheme for load balancing is a direct response to
a case study for computation caching in scientific
applications~\cite{jenkins:ipdsw17-mochi}. In that work the authors motivated
the need for a flexible load balancing \emph{microservice} to efficiently scale
a memoization microservice. Our work is also heavily influenced by the
Malacology project~\cite{sevilla:eurosys17-malacology} which seeks to provide
fundamental services from within the storage system ({\it e.g.}, consensus) to
the application.

State-of-the-art distributed file systems partition write-heavy workloads and
replicate read-heavy workloads, similar to the approach we are advocating
here.  IndexFS~\cite{ren:sc2014-indexfs} partitions directories and clients
write to different partitions by grabbing leases and caching ancestor metadata
for path traversal. ShardFS takes the replication approach to the extreme by
copying all directory state to all nodes. The Ceph file system
(CephFS)~\cite{weil:sc2004-dyn-metadata, weil:osdi2006-ceph} employs both
techniques to a lesser extent; directories can be replicated or sharded but the
caching and replication policies are controlled with tunable parameters.  These
systems still need to be tuned by hand with {\it ad-hoc} policies designed for
specific applications.  Setting policies for migrations is arguably more
difficult than adding the migration mechanisms themselves.  For example,
IndexFS/CephFS use the GIGA+~\cite{patil:fast2011-giga} technique for
partitioning directories at a \emph{predefined} threshold. Mantle makes headway
in this space by providing a framework for exploring these policies, but does
not attempt anything more sophisticated (e.g., machine learning) to create
these policies. 

% ml and autotuning
Auto-tuning is a well-known technique used in
HPC~\cite{behzad:sc2013-autotuning, behzad:techreport2014-io-autotuning}, big
data systems systems~\cite{herodotou_starfish_2011}, and
databases~\cite{schnaitter_index_2009}.  Like our work, these systems focus on
the physical design of the storage ({\it e.g.} cache size) but since we focused
on a relatively small set of parameters (cache size, migration thresholds), we
did not need anything as sophisticated as the genetic algorithm used
in~\cite{behzad:sc2013-autotuning}.  We cannot drop these techniques into
ParSplice because the magnitude and speed of the workload hotspots/flash crowds
makes existing approaches less applicable. 

Our plan is to use MDHIM~\cite{greenberg:hotstorage2015-mdhim} as our back-end
key-value store because it was designed for HPC and has the proper mechanisms
for migration already implemented.  

\section{Future Work}

This lays the foundation for future work, where we will focus on formalizing a
collection of general data management policies that can be used across domains
and services. The value of such a collection eases the burden of policy
development and paves the way for solutions that remove the administrator from
the development cycle, such as (1) adaptable policies that automatically switch
to new strategies when the current strategy behaves poorly ({e.g.}, thrashing,
making no progress, etc.), and (2) policy generation, where new policies are
constructed automatically by examining the collection of existing policies.
Such work is made possible with Mantle's ability to dynamically change
policies.

\section{Conclusion}

Data management encompasses a wide range of techniques that vary by domain and
service. Yet, the techniques require policies that shape the decision making
and finding the best policies is a difficult, multi-dimensional problem. We
observe that many of the primitives and resulting strategies have enough in
common that they can be expressed with similar semantics. We present a data
management language and policy engine, called Mantle that is general enough to
express complicated, dynamic policies for two different domains and services.
Rather than attempting to construct a single, complex load balancing policy
that works for a variety of scenarios, we instead use the Mantle framework to
enable software-defined storage systems to flexibly change policies as the
workload changes over time.  In our analysis of the ParSplice key-value
workload we have detected clear workload regimes that are sensitive to the
initial conditions and the scale and duration of the simulation. We have also
demonstrated that changing load balancing policies at runtime in response to
the current workload is an effective mechanism to providing better load
distribution.  Finally, we have demonstrated that Mantle is flexible enough to
support domain-specific knowledge, which lays the groundwork for future work in
adaptable policies and policy generation.

